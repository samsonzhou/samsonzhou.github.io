\documentclass[11  pt]{article} 
\usepackage[lmargin=1in,rmargin=1.75in,bmargin=1in,tmargin=1in]{geometry}  


% For hyperlinking everything
\usepackage{hyperref}
\hypersetup{
	colorlinks=true, %set true if you want colored links
	linktoc=all,     %set to all if you want both sections and subsections linked
	linkcolor=blue,  %choose some color if you want links to stand out
}


\usepackage[latin1]{inputenc}
\usepackage{amsmath}
\usepackage{mathrsfs}  
\usepackage{amsfonts}
\usepackage{amssymb}
\usepackage{graphicx}
\usepackage{subfig}
\usepackage{caption}
\usepackage{algorithm}
%\usepackage{algcompatible}
%\usepackage{algorithmicx}
\usepackage{algpseudocode}

\usepackage{titlesec}
\titleformat{\section}{\fontfamily{lmss}\fontsize{14}{15}\bfseries}{\thesection}{1em}{}
\titleformat{\subsection}{\fontfamily{lmss}\fontsize{12}{15}\bfseries}{\thesubsection}{1em}{}




\usepackage{amsthm}

\newtheoremstyle{noit}
{10pt}% <Space above>
{10pt}% <Space below>
{}% <Body font>
{}% <Indent amount>
{\bfseries}% <Theorem head font>
{.}% <Punctuation after theorem head>
{.5em}% <Space after theorem headi>
{}% <Theorem head spec (can be left empty, meaning `normal')>

\newtheoremstyle{example}
{10pt}% <Space above>
{10pt}% <Space below>
{}% <Body font>
{20pt}% <Indent amount>
{\bfseries}% <Theorem head font>
{.}% <Punctuation after theorem head>
{.5em}% <Space after theorem headi>
{}% <Theorem head spec (can be left empty, meaning `normal')>


\newtheoremstyle{indented}{20pt}{20pt}{\addtolength{\leftskip}{2.5em}}{}{\bfseries}{.}{.5em}{}


\newtheorem{theorem}{Theorem}
\numberwithin{theorem}{section}
\newtheorem{lemma}[theorem]{Lemma}
\newtheorem{corollary}[theorem]{Corollary}
\newtheorem{observation}{Observation}
%\numberwithin{observation}{section}
%\numberwithin{definition}{section}
\newtheorem{conjecture}{Conjecture}
\newtheorem{Qu}{Question}
\newcommand{\QU}{\begin{Qu}\normalfont}

\theoremstyle{noit}
\newtheorem{fact}{Fact}
\newtheorem{definition}{Definition}

\theoremstyle{indented}
\newtheorem{example}{Example}

\theoremstyle{indented}
\newtheorem{problem}{Problem}


%\newenvironment{proof}{\noindent{\bf Proof:} \hspace*{1em}}{
%    \hspace*{\fill} $\Box$ }
%\newenvironment{proof_of}[1]{\noindent {\bf Proof of #1:}
%    \hspace*{1em} }{\hspace*{\fill} $\Box$ }
%\newenvironment{proof_claim}{\begin{quotation} \noindent}{
%    \hspace*{\fill} $\diamond$ \end{quotation}}
\newcommand{\vs}[1]{\vspace{#1}}

\newcommand{\lecture}[2]{
 \noindent
\begin{center}
	\framebox{
		\vbox{
			\hbox to 5.78in { {\bf CSCE 411: Design and Analysis of Algorithms} \hfill  }
			\vspace{2mm}
			\hbox to 5.78in { {\Large \hfill Lecture #1\hfill} }
			\vspace{2mm}
			\hbox to 5.78in { {\it Date: #2 \hfill Nate Veldt, updated by Samson Zhou} }
		}
	}
\end{center}
\vspace*{4mm}
}


\newcommand{\hw}[2]{
	\noindent
	\begin{center}
		\framebox{
			\vbox{
				\hbox to 5.78in { {\bf CSCE 411: Design and Analysis of Algorithms} \hfill  }
				\vspace{2mm}
				\hbox to 5.78in { {\Large \hfill Homework #1\hfill} }
				\vspace{2mm}
				\hbox to 5.78in { {\it Due date: #2 \hfil} }
			}
		}
	\end{center}
	\vspace*{4mm}
}



\newcommand{\under}[1]{\underline{\hspace{#1}}}
\setlength{\parindent}{0em}

%\usepackage[tagged]{accessibility}

% Graph terms
\newcommand{\vol}{\textbf{vol}}
\newcommand{\cut}{\textbf{cut}}


% Matrices
\newcommand{\mA}{\textbf{A}}
\newcommand{\mB}{\textbf{B}}

% vectors
\newcommand{\ve}{\textbf{e}}
\newcommand{\vx}{\textbf{x}}


% Other
\newcommand{\calN}{\mathcal{N}}

\usepackage{mathtools}
\DeclarePairedDelimiter\ceil{\lceil}{\rceil}
\DeclarePairedDelimiter\floor{\lfloor}{\rfloor}


\newcommand*{\aitem}{ \item[{\includegraphics[width=0.8cm,height=0.5cm]{figures/A}} ]  }
\newcommand*{\bitem}{ \item[{\includegraphics[width=0.8cm,height=0.5cm]{figures/B}} ]  }
\newcommand*{\citem}{ \item[{\includegraphics[width=0.8cm,height=0.5cm]{figures/C}} ]  }
\newcommand*{\ditem}{ \item[{\includegraphics[width=0.8cm,height=0.5cm]{figures/D}} ]  }
\newcommand*{\eitem}{ \item[{\includegraphics[width=0.8cm,height=0.5cm]{figures/E}} ]  }
\newcommand*{\fitem}{ \item[{\includegraphics[width=0.8cm,height=0.5cm]{figures/F}} ]  }


\newcommand{\hide}[1]{\underline{\phantom{#1 #1}}}

\usepackage{setspace}

\onehalfspacing
\newcommand{\tta}{\tt{a}}
\newcommand{\ttb}{\tt{b}}
\newcommand{\ttc}{\tt{c}}
\newcommand{\tte}{\tt{e}}
\newcommand{\ttd}{\tt{d}}
\newcommand{\ttf}{\tt{f}}
\begin{document}
	
	
	\lecture{6: Greedy Algorithms}{February 5, 2025}
	
	\paragraph{Course Logistics}
	
	\begin{itemize}
		\item Greedy algorithms and amortized analysis this week: Chapter 16
		\item Homework 3 is posted, due this Friday
	\end{itemize}
	

\section{Binary Codes}

%	In binary code compression, the goal is to compress a string of characters efficiently using 0's and 1's.
\begin{itemize}
	\item An \hide{alphabet  } is a set of characters, e.g., $C = \{\tt{a},\tt{b}, \tt{c},\tt{d} ,\tt{e}, \ttf\}$ \\
	\item A \emph{binary code} for $C$ is\\ %a way to represent each $c \in C$ using a binary string.
\end{itemize}

Consider three possible codes for $C = \{\tt{a},\tt{b}, \tt{c},\tt{d} ,\tt{e}, \ttf\}$.
\begin{equation}
	\label{tab1}
	\begin{tabular}{c c c c c c c}
		Character: &  \tta & \ttb & \ttc & \ttd & \tte & \ttf \\
		\hline
		example code 1: & 000 & 001 & 010 &  011 & 100 & 101 \\
		example code 2: &0 & 101 & 100 & 111 & 1101 & 1100 \\
		example code 3: &0 & 1 & 10 & 01 & 11 & 00 \\
		\hline
	\end{tabular} \\
\end{equation}

Types of codes:
\begin{itemize}
	\item \textbf{Fixed length code}: %all codes have the same length
	\item \textbf{Variable-length code}: %codes can have different lengths (but are still unique)
	\item \textbf{Prefix code}: the codeword for every $c \in C$ is % \emph{not} the start of any other codeword
\end{itemize}

\QU 
Which of the codes in Table~\ref{tab1} is a prefix code?
\begin{itemize}
	\aitem Example code 1
	\bitem Example code 2
	\citem Example code 3
	\ditem Example codes 1 and 2
	\eitem Example codes 1, 2, and 3
\end{itemize}
\end{Qu}

\newpage 

\subsection{Encoding and Decoding}
Consider a string of characters from the alphabet $C = \{\tt{a},\tt{b}, \tt{c},\tt{d} ,\tt{e}, \ttf\}$ where the frequency of each character is given in the following table and we have two codes.
\begin{equation}
\begin{tabular}{c c c c c c c}
	Character: &  \tta & \ttb & \ttc & \ttd & \tte & \ttf \\
	\hline
	Frequency & 0.45 & 0.13 & 0.12 & 0.16 & 0.09 & 0.05 \\
	FLC: & 000 & 001 & 010 &  011 & 100 & 101 \\
	VLC: &0 & 101 & 100 & 111 & 1101 & 1100 \\
	\hline
\end{tabular} 
\end{equation}
\newpage


\subsection{The Optimal Prefix Code Problem}

\paragraph{The optimal prefix code problem} Given an alphabet $C$ and a frequency $\mathit{c.freq}$ for each $c \in C$ in a given string $s$,  \\ \\ %find the prefix code that represents $s$ using a minimum number of bits. \\


\paragraph{Representing a prefix code as a tree}
Every prefix code for a string $s$ can be represented as a binary tree $T$ where 
\begin{itemize}
\item Left branches are associated with %`0'
\item Right branches are associated with %`1'
\item Each leaf corresponds to a character $c \in C$, whose binary code is given by \\
%the bits in the path from the root to the leaf
\item Each node is associated with the frequency of characters in its subtree
\end{itemize}

\paragraph{Example}
\begin{tabular}{c c c c c c c}
Character: &  \tta & \ttb & \ttc & \ttd & \tte & \ttf \\
\hline
Frequency & 0.45 & 0.13 & 0.12 & 0.16 & 0.09 & 0.05 \\
VLC: &0 & 101 & 100 & 111 & 1101 & 1100 \\
\hline
\end{tabular} 

\newpage

The cost of the tree is give by \\
\begin{equation*}
B(T) = \sum_{c \in C} \\%\mathit{c.freq} \cdot d_T(c)
\end{equation*}
where $d_T(c)$ is the depth of $c$ in the tree $T$. 

\QU
True or false: every binary tree (with $\ell$ leaves) gives a valid prefix code (for an alphabet with $\ell$ characters).
\begin{itemize}
\aitem True
\bitem False
\end{itemize}
\end{Qu}


\newpage

\begin{theorem}
%Any binary tree with $|C|$ leaves gives a valid prefix code for $C$.
\end{theorem}
\vs{1.5cm}
\textit{Proof.}

\vfill



\paragraph{Optimal prefix code problem:} 

\vs{1cm}
%Find the binary tree representation %with minimum cost. 


\paragraph{Nice properties about the optimal tree}
In the optimal tree representation,
\begin{itemize}
\itemsep = 3em
\item %There are $|C|$ leaves
\item %Each node in the tree has a sibling
\end{itemize}



\newpage

\subsection{Huffman Codes}
Huffman Code: a prefix code obtained by greedily building a tree representation for $C$

%\textbf{Input:} Alphabet $C$, for each $c \in C$, a frequency $c.freq$.\\
%\textbf{Output:} A binary tree representing a prefix code

\paragraph{Huffman Code Tree Process}
\begin{itemize}
	\item Associate each character in $C$ with a node, labeled by its frequency
	\item Identify two nodes $x$ and $y$ in $C$ with the \hide{smallest frequencies}
	\item Create new node $z$ with frequency \hide{$z.freq = x.freq + y.freq$.}
	\item Make $z.left = x$, and $z.right = y$.
	\item Repeat the above procedure on the alphabet obtained by \hide{replacing $x$ and }\\
\end{itemize}
Observe: $x$ and $y$ will be siblings in $T$ at \hide{a maximum depth.}
\paragraph{Activity} Create a Huffman code for:
\begin{tabular}{c c c c c c c}
	Character: &  \tta & \ttb & \ttc & \ttd & \tte & \ttf \\
	\hline
	Frequency & 0.45 & 0.13 & 0.12 & 0.16 & 0.09 & 0.05 
\end{tabular} 

\newpage

\subsection{Code and Illustration}
\begin{algorithm}[t]
	\textsc{HuffmanCode}($C$)
	\begin{algorithmic}
		\State $n =|C|$
		\State $T \leftarrow$ empty graph 
		\State $Q \leftarrow \emptyset$ 
		\For{$c$ in $C$}
		\State Insert $c$ with value $c.freq$ into $T$ and $Q$
		\EndFor
		\For{$i = 1$ to $n-1$}
		\State $x = $ % \textsc{ExtractMin}(Q)$
		\State $y = $ %\textsc{ExtractMin}(Q)$
		\State create node $z$
		\State 
		\State 
		\State 
		\State
		\State 
		\State
		\State 
		%			\State Insert $z$ into $T$
		%			\State Insert $z$ into $Q$ with frequency $z.val$
		\EndFor
		\State Return $T$
	\end{algorithmic}
\end{algorithm}
\newpage

\subsection{Runtime Analysis}
Given a set of objects $C$, with values $c.val$ for $c \in C$ and $n = |C|$, a binary min-heap for $C$ is a binary tree such that
\begin{itemize}
	\item All levels %(except maybe the last) are complete
	\item The value of a node is % $\leq$ the value of its descendants.
\end{itemize}
It has the following operations:
\begin{itemize}
	\item $\textsc{BuildBinMinHeap}(C)$:  \\% builds heap $Q$ from elements in $C$, takes $O(n)$ time.
	\item $\textsc{ExtractMin}(Q)$: \\
	
	%removes minimum element. Takes $O(\log n)$ time.
	\item $\textsc{Insert}(Q,z,z.freq)$: \\ %adds an element to $Q$. Takes $O(\log n)$ time.
\end{itemize}

\QU
Assume we use a binary min heap to store and update $Q$ in the pseudocode above. Then what is the overall runtime of creating a Huffman code, in terms of $n = |C|$?
\begin{itemize}
	\aitem $O(\log n)$
	\bitem $O(n)$
	\citem $O(n \log n)$
	\ditem $O(n^2)$
\end{itemize}
\end{Qu}


\pagebreak

\subsection{Optimality}
It turns out that a Huffman code optimally solves the prefix code problem. The argument hinges on the following lemma.

\begin{lemma}
Let $C$ be an alphabet where $c.freq$ is the frequency of $c \in C$. Let $x$ and $y$ be the two characters in $C$ that have the lowest frequencies. \\

Then, there exists an optimal prefix code for $C$ in which $x$ and $y$ have the same length code words and only differ in the last bit. \\

Equivalently: \\% there exists a Huffman code tree in which $x$ and $y$ are sibling leaves at maximum depth.
\end{lemma}



\newpage 
\paragraph{Why does this imply optimality?}
Consider a slightly different (but equivalent) approach to defining a Huffman code:
\begin{enumerate}
\item Associate each character with a node, labeled by its frequency
\item Identify the two nodes $x$ and $y$ with the smallest frequencies
\item Create new node $z$ with frequency $z.freq = x.freq + y.freq$. 
\item  %Let $T'$ be an optimal tree for alphabet $C'$ obtained by replacing $x$ and $y$ with $z$
\item Create tree $T$ from $T'$ by making $z.left = x$, and $z.right = y$.
\end{enumerate}

\paragraph{Claim}
This produces an optimal tree $T$. 

\textit{Proof.}
Note that the cost of the tree $T$ is 
\begin{equation}
\label{this}
B(T) = B(T') + x.freq + y.freq
\end{equation}

\vs{2cm}

%	Why? Because instead of a node $z$ of depth $d$, we have nodes $x$ and $y$ at depth $d+1$, so for two characters, the codeword is 1 longer. \\ 

We prove the result by contradiction: suppose that $T$ is not optimal. Then %there exists some tree $R$ that is better,

%$$B(R) < B(T)$$

\vs{2cm}

We can assume that $x$ and $y$ are at maximum depth in $R$. \\

%	
Let $R'$ be the tree obtained by taking $R$ and \\ %replacing $x$ and $y$ with a merged node $z$. 

Similar to~\eqref{this} we have that:  \\


We then get an impossible sequence of inequalities:
\begin{align*}
B(R') &= B(R) - x.freq - y.freq \\
& < B(T) - x.freq - y.freq \\
& = B(T') 
\end{align*}

%	This says that $R'$ is a strictly better tree for $C'$, but we said $T'$ was optimal for $C'$. Contradiction!


\end{document}
