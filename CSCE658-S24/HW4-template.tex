\documentclass[11pt]{article}
\pdfoutput=1
\usepackage[T1]{fontenc}
\usepackage{lmodern}
\usepackage[protrusion=true,expansion=true]{microtype}
\usepackage{amsmath,amssymb,amsfonts,amsthm}
\usepackage{subcaption}
\usepackage{graphicx}
\usepackage{fullpage}
\usepackage{setspace}
\usepackage[backref=page]{hyperref}
\usepackage{color}
\usepackage{wrapfig}
\usepackage{tikz}
\usetikzlibrary{decorations.pathreplacing}
\usepackage{algorithm}
\usepackage[noend]{algpseudocode}
\usepackage[framemethod=tikz]{mdframed}
\usepackage{xspace}
\usepackage{pgfplots}
\usepackage{framed}

\newcommand{\Ex}[1]{\ensuremath{\mathbb{E}\left[#1\right]}}
\newcommand{\Var}[1]{\ensuremath{\text{Var}\left[#1\right]}}
\newcommand{\PPr}[1]{\ensuremath{\mathbf{Pr}\left[#1\right]}}
\newcommand{\eps}{\varepsilon}
\newcommand{\tail}{\textsc{Tail}}
\newcommand{\CountSketch}{\textsc{CountSketch}}

\allowdisplaybreaks

\begin{document}
\begin{center}
{\Large\textsc{CSCE 658: Randomized Algorithms -- Spring 2024 \\ 
Problem Set 4}}
\vskip 0.1in
%%%%%%%%%%%%%%%%%%%%%%%%%%%%%%%%%%%%%%%%%%%%%%%%%%%%%%%%%%%%%%%%%%%%
%%%%%%%%%%%        Toggle the next two lines      %%%%%%%%%%%%%%%%%%
%%%%%%%%%%%%%%%%%%%%%%%%%%%%%%%%%%%%%%%%%%%%%%%%%%%%%%%%%%%%%%%%%%%% 
%YOUR NAME(S) HERE
Due: Tuesday, April 23, 2024, 5:00 pm CT
\end{center}


%%%%%%%%%%%%%%%%%%%%%%%%%%%%%%%%%%%%%%%%%%%%%%%%%%%%%%%%%%%%%%%%%%%%
%%%%%%%%%%%            Problem 1 Begins Here      %%%%%%%%%%%%%%%%%%
%%%%%%%%%%%%%%%%%%%%%%%%%%%%%%%%%%%%%%%%%%%%%%%%%%%%%%%%%%%%%%%%%%%%
\noindent
\textbf{Problem 1.} (30 points total)
\begin{enumerate}
\item (10 points)
Prove that every graph on $m$ edges has a subgraph on at least $\frac{m}{2}$ edges that is bipartite.
\end{enumerate}

\noindent\textbf{Solution:}

\begin{enumerate}
\setcounter{enumi}{1}
\item (10 points)
An independent set of a graph is a subset of vertices that are not connected by an edge in the graph. 
Prove that for any graph with $n$ vertices, and $m\ge\frac{n}{2}$ edges, there exists an independent set of
size at least $\frac{n^2}{4m}$. 
\end{enumerate}

\noindent\textbf{Solution:}

\begin{enumerate}
\setcounter{enumi}{2}
\item (10 points)
Prove that for every matrix $A\in\{0,1\}^{n\times n}$, there exists a vector $b\in\{-1,+1\}^n$ such that there is some entry of $Ab\in\mathbb{R}^n$ with magnitude at least $2\sqrt{n\log n}$. 
\end{enumerate}

\noindent\textbf{Solution:}




%%%%%%%%%%%%%%%%%%%%%%%%%%%%%%%%%%%%%%%%%%%%%%%%%%%%%%%%%%%%%%%%%%%%
%%%%%%%%%%%            Problem 2 Begins Here      %%%%%%%%%%%%%%%%%%
%%%%%%%%%%%%%%%%%%%%%%%%%%%%%%%%%%%%%%%%%%%%%%%%%%%%%%%%%%%%%%%%%%%%
\newpage\noindent
\textbf{Problem 2.} (30 points total)
\begin{enumerate}
\item (15 points)
Suppose there are $p$ packets that need to be routed over a network of links. 
Each packet $i\in[p]$ must pick a route from a set $R_i$ of $r$ different possible routes to be sent. 
Multiple routes can share the same link, but a link only has capacity to support the routing of a single packet. 
Suppose that for all $i\neq j$ and any route $R\in R_i$, there are at most $c$ other routes $R'\in R_j$ that share a link with $R$. 
Prove that if $r\ge 8pc$, then there exists a possible routing of all $p$ packets where no link exceeds its capacity.
\end{enumerate}

\noindent\textbf{Solution:}

\begin{enumerate}
\setcounter{enumi}{1}
\item (15 points)
Let $G$ be an undirected graph and suppose each vertex $v$ has a set $C(v)$ of colors. 
A proper list coloring of of the graph assigns each vertex $v\in V$ a color from its set $C(v)$ while ensuring that no edges have two vertices with the same color. 
Suppose $|C(v)|\ge 10q$ and for all $v\in V$ and $c\in S(v)$, there are most $q$ neighbors $u$ of $v$ that contain $c$ in $C(u)$. 
Prove that there exists a proper list coloring of the graph. 
\end{enumerate}

\noindent\textbf{Solution:}



%%%%%%%%%%%%%%%%%%%%%%%%%%%%%%%%%%%%%%%%%%%%%%%%%%%%%%%%%%%%%%%%%%%%
%%%%%%%%%%%            Problem 3 Begins Here      %%%%%%%%%%%%%%%%%%
%%%%%%%%%%%%%%%%%%%%%%%%%%%%%%%%%%%%%%%%%%%%%%%%%%%%%%%%%%%%%%%%%%%%
\newpage\noindent
\textbf{Problem 3.} (30 points total)
\begin{enumerate}
\item (10 points)
Give an example, with proof, of a primal-dual pair of linear programs, each with at most three variables and three constraints in addition to the non-negativity constraints, such that neither program is feasible. 
\end{enumerate}

\noindent\textbf{Solution:}

\begin{enumerate}
\setcounter{enumi}{1}
\item (10 points)
Write the linear program for the best fit line with $L_1$ error, i.e., values $(a,b,c)$ that minimizes 
\[\sum_{i=1}^n|ax_i+by_i-c|.\]
\end{enumerate}

\noindent\textbf{Solution:}

\begin{enumerate}
\setcounter{enumi}{2}
\item (10 points)
Write the linear program for the best fit line with $L_\infty$ error, i.e., values $(a,b,c)$ that minimizes 
\[\max_{i\in[n]}|ax_i+by_i-c|.\] 
\end{enumerate}

\noindent\textbf{Solution:}


%%%%%%%%%%%%%%%%%%%%%%%%%%%%%%%%%%%%%%%%%%%%%%%%%%%%%%%%%%%%%%%%%%%%
%%%%%%%%%%%            Problem 4 Begins Here      %%%%%%%%%%%%%%%%%%
%%%%%%%%%%%%%%%%%%%%%%%%%%%%%%%%%%%%%%%%%%%%%%%%%%%%%%%%%%%%%%%%%%%%
\newpage\noindent
\textbf{Problem 4.} (30 points total)
\begin{enumerate}
\item (5 points)
Describe the implementation of randomized response for $\eps$-differential privacy. 
Prove its correctness.
\end{enumerate}

\noindent\textbf{Solution:}

\begin{enumerate}
\setcounter{enumi}{1}
\item (5 points)
Use the probability density function of the Laplace distribution to prove that if $X\sim\text{Lap}(b)$, then $\PPr{|X|>4Cb\log n}\le\frac{1}{n^C}$ for any constant $C>0$.  
\end{enumerate}

\noindent\textbf{Solution:}

\begin{enumerate}
\setcounter{enumi}{2}
\item (10 points)
Suppose that are given a database $x_1,\ldots,x_n$ of counts, so that $x_i\in\mathbb{Z}$ for all $i\in[n]$. 
To privately release the index of the item with the largest value, i.e., $\text{argmax}_{i\in[n]} x_i$, we first add independent Laplace noise $\text{Lap}\left(\frac{1}{\eps}\right)$ to each value $x_i$ to acquire a value $y_i$. 
We then output the index of the noisy item with the largest value, i.e., $\text{argmax}_{i\in[n]} y_i$. 
Show with proof that the resulting protocol is $\eps$-differentially private. 
Analyze the correctness/error of the protocol. 
\end{enumerate}

\noindent\textbf{Solution:}

\begin{enumerate}
\setcounter{enumi}{3}
\item (10 points)
Suppose that are given a database $x_1,\ldots,x_n$ of counts, so that $x_i\in\mathbb{Z}$ for all $i\in[n]$. 
Suppose that we release $i\in[n]$ with probability proportional to $\exp(s(x_i))$, where $s(x_i)=x_i-\max_{j\in[n]} x_j$. 
Show with proof that the resulting protocol is $\eps$-differentially private. 
Analyze the correctness/error of the protocol. 
\end{enumerate}

\noindent\textbf{Solution:}
\end{document}