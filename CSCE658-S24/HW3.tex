\documentclass[11pt]{article}
\pdfoutput=1
\usepackage[T1]{fontenc}
\usepackage{lmodern}
\usepackage[protrusion=true,expansion=true]{microtype}
\usepackage{amsmath,amssymb,amsfonts,amsthm}
\usepackage{subcaption}
\usepackage{graphicx}
\usepackage{fullpage}
\usepackage{setspace}
\usepackage[backref=page]{hyperref}
\usepackage{color}
\usepackage{wrapfig}
\usepackage{tikz}
\usetikzlibrary{decorations.pathreplacing}
\usepackage{algorithm}
\usepackage[noend]{algpseudocode}
\usepackage[framemethod=tikz]{mdframed}
\usepackage{xspace}
\usepackage{pgfplots}
\usepackage{framed}

\newcommand{\Ex}[1]{\ensuremath{\mathbb{E}\left[#1\right]}}
\newcommand{\Var}[1]{\ensuremath{\text{Var}\left[#1\right]}}
\newcommand{\PPr}[1]{\ensuremath{\mathbf{Pr}\left[#1\right]}}

\allowdisplaybreaks

\begin{document}
\begin{center}
{\Large\textsc{CSCE 658: Randomized Algorithms -- Spring 2024 \\ 
Problem Set 3}}
\vskip 0.1in
Due: Tuesday, February 20, 2024, 5:00 pm CT
\end{center}

\noindent
\textbf{Problem 1.} (30 points total)
CountSketch tail bounds. 
\end{enumerate}




\vskip 0.2in\noindent
\textbf{Problem 2.} (30 points total)
AMS Sketch for $F_p$



%\vskip 0.1in\noindent
%\begin{center}
%(Continued on next page)
%\end{center}
%\newpage
\vskip 0.2in\noindent
\textbf{Problem 3.} (30 points total)
Oblivious routing



\vskip 0.2in\noindent
\noindent
\textbf{Problem 4.} (30 points total)




\end{document}